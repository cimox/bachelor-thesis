%!TEX root = ./main.tex

% This file is part of the i10 thesis template developed and used by the
% Media Computing Group at RWTH Aachen University.
% The current version of this template can be obtained at
% <http://www.media.informatik.rwth-aachen.de/karrer.html>.

\loadgeometry{myAbstract}

\chapter*{Abstract\markboth{Abstract}{Abstract}}
\addcontentsline{toc}{chapter}{\protect\numberline{}Abstract}
\label{abstract}

%english version
Nowadays we can see Big Data processing and analysis in many domains. As a amounts of data growing, more people are focusing on this problem. The most affected domains are social media websites like Facebook or Twitter. A data from such a sources are streaming in huge amounts and changing in real-time, called data streams. We want to process and analyze data streams in real-time to provide users personalized and valuable outputs. The most common approach to handle data streams is map-reduce paradigm, e.g. batch data processing. Proposed methods are not meeting our requirement to process data streams in real-time. To achieve these requirements, we need use different approach called data stream processing which is built on Lambda Architecture.\\
Processing and analysis of big data streams is a complex task, because we need to provide low-latency, scalable and fault-tolerant solution. In our project, we analyze existing solutions and frameworks to analyze data streams. We provide verification of its characteristics in different kind of tasks. Accordingly to this, we propose a application for processing and analyzing big data streams (e.g. Twitter data stream), which allows users to get valuable outputs changing in real-time.\\
\todo{upravit anglicku podla slovenskej. V slovenskej su male upravy...- az po revizii slovenskej}
\emptydoublepage
\chapter*{Abstrakt\markboth{Abstrakt}{Abstrakt}}
\addcontentsline{toc}{chapter}{\protect\numberline{}Abstrakt}
\label{abstrakt
%slovak version
Dnes sa stretávame so spracovaním a anlýzou veľkého objemu dát v mnohých oblastiach. S narastajúcim objemom dát rastie aj záujem o tento problem spracovania veľkého objemu dát. Toto najviac postihuje oblasť sociálnych medií, napríklad Facebook, či Twitter. Dáta z takýchto zdrojov prúdia v obrovkých množstvách, pričom dáta sa v čase menia, takéto prúdiace dáta nazývame jednoducho prúd dát. Prúdy dát chceme spracovať a analyzovat v reálnom čase, aby sme mohli ponúknuť používateľom personalizovaný a hodnotný výstup. Najčastejší prístup spracovania prúdu dát je map-reduce paradigma, napríklad dávkové spracovanie. Táto metóda, ale nespĺňa naše požiadavky na spracovanie dát v reálnom čase. Na dosiahnutie našej požiadavky potrebujeme iný prístup, ktorý nazývame spracovanie prúdu dát. Metóda takéhoto prístupu je postavená na Lambda architektúre.
Spracovanie a analýza veľkých prúdov dát je komplexný problém, pretože prúd musí byť spracovaný s nízkou odozvou, riešenie musí byť odolné proti chybám a  škálovateľné. V našej práci budeme analyzovať existujúce riešenia and rámce na analýzu prúdu dát. Poskytujeme overenie charakteristík týchto riešení v roznych problémoch, ktoré si vyžadujú spracovanie prúdu dát v reálnom čase. Na základe tohto poskytujeme aplikáciu, ktorá spracuje a analyzuje veľký prúd dát (napríklad prúd dát Twittru), ktorá umožní používateľom získať hodnotný výstup, ktorý sa mení v reálnom čase.\\
\todo{zrevidovat slovensku verziu}
%Slovenska verzia.
\emptydoublepage
\loadgeometry{myText}
