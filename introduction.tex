%!TEX root = ./main.tex

\chapter{Úvod}
V ostatnom čase si téma spracovania prúdu údajov v kontexte veľkého objemu dát (angl. Big Data) vyžaduje stále väčšiu pozornosť. Čím ďalej, sa vo viacerých doménach stretávame s problémom spracovania narastajúceho objemu údajov, ktoré sú rozmanité. Tradičné prístupy obchodných informácií (angl. business intelligence) nie sú postačujúce pri riešení týchto problémov. \citep{liu2014survey}. 
\\
Spracovanie Big Data sa stáva súčasťou stále väčšieho množstva odvetví. Či už ide o veľkých telekomunikačných operátorov, komerčné podniky, vyhľadávače alebo sociálne média. 
Značným zdrojom dát v súčasnosti sú senzory nachádzajúce sa v zariadeniach, ktoré sú súčasťou bežného života dnes. 
Inteligentné telefóny a hodinky, športové náramky, či vo všeobecnosti internet vecí (angl. Internet of Things\footnote{IoT council: http://www.theinternetofthings.eu/}) zmenšuje priepasť medzi svetom fyzických zariadení a prepojení s internetom. 
Ako príklad masívnosti Big Data môžeme uviesť šesť hodinový medzištátny let Boeingu 737 z New Yorku do Los Angeles, ktorý vygeneruje počas letu približne 240 terabajtov dát \citep{boeing}. Takéto rýchlo vznikajúce dáta, ktoré prúdia vo veľkých objemoch nazývame \textit{prúd údajov} (angl. data stream). 
\\
Preto stúpa motivácia a význam vybudovať infraštruktúru, ktorá bude schopná spracovať takýto masívny objem prúdiadich dáť v reálnom čase. Existuje veľa aplikácií, ktorých správne fungovanie môže byť kritické vzhľadom na správnosť výsledku zo súvislého a nekonečného prúdu dát. Ako príklad môžme uviesť letové údaje zo senzorov lietadla Boeing 737 alebo analýzu trendov na sociálnej sieti Twitter\footnote{http://www.twitter.com/}. Keby sme takýto prúd spracovali tradičným prístupom spracovania Big Data, dávkovým spracovaním, mohlo by to mať kritický dopad na výsledky aplikácie a súvisiace následky. V prípade Boeingu 737 je jasné, že systém musí poskytnúť výsledky v takmer reálnom čase, inak to môže mať katastrofické následky. Pri analýze trendov na sociálnej sieti Twitter je veľmi pravdepodobné\citep{mathioudakis2010twittermonitor}, že sa v spracovanej dávke dát stratí trend, ktorý je aktuálny iba pre krátky časový interval. Preto je dáta potrebné spracovávať okamžite po vstupe do aplikácie. 
\\
	Na úvod je potrebné správne vysvetliť a pochopiť pojem spracovanie v \textit{reálnom čase}. V mnohých systémoch je kritické spracovanie a odozva systému v reálnom čase, avšak chápanie tohto pojmu sa výrazne líši v závislosti od systému, či aplikácie. Aplikácie spĺňajúce kritérium spracovania v reálnom čase považujeme vtedy, ak tieto kritéria spĺňajú pre požiadavky reálneho sveta. 
	%Väčšinou nazývame aplikácie, ktoré spĺňajú kritéria spracovania v reálnom čase a garantujú doručenie správ, ak dokážu tieto kritéria splniť pre požiadavky reálneho sveta. 
	Napríklad vysvietenie a vychýlenie fotónu na zobrazovej jednotke osciloskopu určite spĺňa kritériá reálneho sveta. V našej práci zameranej na analýzu prúdu dát chápeme spracovanie v reálnom čase ako metódu kontinuálneho spracovania prúdiacich informácií. Presné časové ohraničenie na spracovanie nie je stanovené, zásadné kritrérium je garancia spracovania všetkých správ, ktoré vstupujú do systému. V nami rozoberanom probléme bude tiež kritické časové ohraničenie v takmer reálnom čase, rádovo desiatky milisekúnd. 
\\
%In our project we analyze existing methods and frameworks for processing data streams. We are building software architecture to process data streams with near-real time latency. This architecture provides valuable outputs by applying filters, aggregations and groupings on a stream, thus extracting valuable information. Outputs are based on user’s queries to data streams, where near-real time processing is essential. 
%Method we use is a combination of pipelines and filters, and construction of acyclic oriented graph (called also topology) with self-adjusting amount of processing elements. This topology is designed to process high-volume data streams, it is highly horizontally scalable and fault-tolerant. We identify common bottlenecks in topology and will try to propose methods and solutions to avoid them.
%To evaluate the properties of proposed topologies in processing of streaming data in near-real time we use sources of data such as Twitter and static dataset collection of tweets (messages) from Twitter. We will evaluate the performance of suggested topology set up using Storm framework [4].
V našej práci analyzujeme existujúce metódy a programovacie rámce na spracovanie prúdiacich údajov. Metodologicky a inkrementálne navrhujeme softvérovú architektúru na spracovanie prúdov v takmer-reálnom čase. Pomocou tejto architektúry extrahujeme informácie z prúdu údajov a poskytujeme hodnotné výstupy na základe používateľských dopytov aplikovaním filtrov, agregácií, či zoskupení. 
\\
Práca je štrukturovaná do štyroch logických celkov, ktoré predstavujú jadro práce. Prvé dve časti hovoria o analýze problému a existujúcich riešeniach tohto problému. V ďalšej časti hovoríme o návrhu našeho riešenia. Posledná časť do hĺbky pojednáva o overovaní a experimentoch nad našim návrhom riešenia. Na konci práce je uvedené zhodnotenie a pohľad do budúcich smerov, zoznam literatúry a prílohy práce.


